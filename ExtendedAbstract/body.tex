%Why should we car?
Additive manufacturing --also known as 3D printing-- has gained popularity thanks the emergence of inexpensive printers and materials~\cite{Oropallo2015}. This situation has attracted the attention of researchers in many fields for improving the manufacturing process~\cite{Gao2015}.

Particularly, the computer graphics community has addressed several problems, like adding mechanical properties of the pieces~\cite{Bacher2014}, \cite{Li2015}, \cite{Panetta2015} or creating models with specific functionality~\cite{Song2015}, \cite{Bacher2014} and analyzing the support structures needed to print~\cite{Vanek2014}, \cite{Dumas2014}, \cite{Hu2016}. One of the most important problems is to ensure viability of the model to print .

%Why si difficult?
In literature there are two approaches to detect problems in \emph{printability} either using mechanical analysis~\cite{Stava2012} or using morphological tools~\cite{Telea2011}. The input model then is altered in order to make it feasible to print. However, the nature of this alterations have been mostly focused on the goal solve mechanical problems~\cite{Lu2014}, \cite{Zhou2013}. The work of~\cite{Echevarria2014}, is one notable exception, however they focused only in a very narrow particular area.

However, one of the preferred uses for 3D printing is for decoration where the looks are as important as other physical properties. The classification of a model using perception has been studied extending the concept of \emph{saliency}~\cite{Wang2015} from the digital image processing literature. Finally, the simplification of a mesh has also been studied in the concept of mesh abstraction~\cite{Mehra2009}, \cite{Yumer2012} and \cite{DeGoes2011}.

%Objective and results
The goal of this study is to develop a method to modify input models in order to archive profitability using perception as a goal. The method will use the morphological analysis to detect regions of the model that are not printable. Then, it will try to locally modify the model using abstraction until the printability is archived. The process would be ideally interactive, but at the same time it will try to minimize the required input form the user, so the system will be an easy to use tool to prepare model to print with minimal visual alterations.
