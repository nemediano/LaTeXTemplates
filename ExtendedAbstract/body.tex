%Why should we care? Intro and problem definition
Additive manufacturing --also known as 3D printing-- has gained recent popularity thanks to the emergence of inexpensive printers and materials~\cite{Oropallo2015}. This situation has attracted the attention of researchers in many fields for improving the manufacturing process~\cite{Gao2015}. However, since the technology is not mature yet, arriving at a successful print still presents a challenge for the non-expert user. One of the most important challenges is to ensure \emph{printability}: the viability of a model to print.

%Previous work
The computer graphics community has addressed several problems in 3D printing; like adding mechanical properties to the models~\cite{Li2015}, \cite{Panetta2015}, creating models with specific functionality~\cite{Song2015}, \cite{Bacher2014} and analyzing the support structures needed to print~\cite{Vanek2014}, \cite{Hu2016}. Currently, there are two approaches for detecting printability problems, either using mechanical analysis~\cite{Stava2012} or using \emph{volumetric representation} with morphological tools~\cite{Telea2011}. Regardless of the approach chosen, the input model is then modified in order to make it feasible to print. However, these alterations have been mostly focused on solving mechanical problems rather than archiving visually pleasant results~\cite{Lu2014}, \cite{Zhou2013}. The work of~\cite{Echevarria2014}, is one notable exception, however it is focused only in the very particular area of hair simulation.

%Key observation - Ah ha momment
One of the preferred uses for 3D printing is for creating decorations, toys or figurines where the looks are important. In computer graphics, the analysis of a model using perception has been studied by extending the concept of \emph{saliency}~\cite{Wang2015} from the digital image processing field, and more recently by the simplification of a mesh in the concept of mesh \emph{abstraction}~\cite{DeGoes2011}. In the works of~\cite{Mehra2009} and \cite{Yumer2012}, the abstraction is constructed by first creating a proxy surface that contains the model using a \emph{volumetric representation} and then making the proxy converge to the model surface by optimization. This parallelism in the representation, suggested that we can use this representation as a guide to archive both printability and perceptual similarity.

%Describe the solution
The goal of this study is to develop a method to modify input models in order to archive printability using a perception metric as a constraint. The method will use the morphological analysis to detect regions of the model that are not printable. In order to make this analysis, a volumetric representation will be created. Then, similarly to the works on abstraction, it will try to locally modify the model using the new representation as a guide until the resulting model can be printed. The process would be ideally interactive, but at the same time it will try to minimize the required input from the user, so the system will be an easy to use tool to prepare models for printing with minimal visual alterations.
