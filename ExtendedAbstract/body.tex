%Why should we care?
Additive manufacturing --also known as 3D printing-- has gained recently popularity thanks the emergence of inexpensive printers and materials~\cite{Oropallo2015}. This situation has attracted the attention of researchers in many fields for improving the manufacturing process~\cite{Gao2015}. Particularly, the computer graphics community has addressed several problems in 3D printing; like adding mechanical properties to the models~\cite{Li2015}, \cite{Panetta2015}, creating models with specific functionality~\cite{Song2015}, \cite{Bacher2014} and analyzing the support structures needed to print~\cite{Vanek2014}, \cite{Hu2016}. 

%What has been done
One of the most important problems is to ensure viability of a model to print. This is a two step process. First, the model is analyzed in order to detect problematic parts; there are two approaches for this task: either using mechanical analysis~\cite{Stava2012} or using volumetric representation with morphological tools~\cite{Telea2011}. Second, the input model is modified in order to make it feasible to print. The nature of this alterations have been mostly focused on solving mechanical problems rather than archive visually pleasant results~\cite{Lu2014}, \cite{Zhou2013}. The work of~\cite{Echevarria2014}, is one notable exception, however it is focused only in the very particular area of hair simulation.

%Key observation
Currently one of the preferred uses for 3D printing is for creating decorations --where the looks are as important as their mechanical properties--. The analysis of a model using perception has been studied by extending the concept of \emph{saliency}~\cite{Wang2015} from the digital image processing field, and more recently by the simplification of a mesh in the concept of mesh \emph{abstraction}~\cite{DeGoes2011}. In the works of~\cite{Mehra2009} and \cite{Yumer2012}, the abstraction is constructed by first creating a proxy surface that contains the model using a volumetric representation and then making the proxy converge to the model surface by optimization.

%Describe the solution
The goal of this study is to develop a method to modify input models in order to archive profitability using a perception metric as constrain. The method will use the morphological analysis to detect regions of the model that are not printable. Then, it will try to locally modify the model using abstraction until the \emph{printability} is archived. The process would be ideally interactive, but at the same time it will try to minimize the required input from the user, so the system will be an --easy to use-- tool to prepare models for printing with minimal visual alterations.
