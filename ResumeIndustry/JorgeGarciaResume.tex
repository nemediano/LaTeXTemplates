%%%%%%%%%%%%%%%%%%%%%%%%%%%%%%%%%%%%%%%%%
% Medium Length Professional CV
% LaTeX Template
% Version 2.0 (8/5/13)
%
% This template has been downloaded from:
% http://www.LaTeXTemplates.com
%
% Original author:
% Trey Hunner (http://www.treyhunner.com/)
%
% Important note:
% This template requires the resume.cls file to be in the same directory as the
% .tex file. The resume.cls file provides the resume style used for structuring the
% document.
%
%%%%%%%%%%%%%%%%%%%%%%%%%%%%%%%%%%%%%%%%%

%----------------------------------------------------------------------------------------
%	PACKAGES AND OTHER DOCUMENT CONFIGURATIONS
%----------------------------------------------------------------------------------------

\documentclass{resume} % Use the custom resume.cls style

% custom links and pdf metadata
\usepackage[
  final,
  unicode,
  colorlinks=true,
  citecolor=black,
  linkcolor=black,
  urlcolor=black,
  plainpages=false,
  urlcolor=black,
  pdfpagelabels=true,
  pdfsubject={Jorge Garcia's Resume},
  pdfauthor={Jorge Antonio García Galicia},
  pdftitle={Resume - Jorge Garcia},    % title
  pdfkeywords={Resume, Computer Science, PhD, Graphics}, % list of keywords
]{hyperref}

\usepackage[letterpaper,left=1cm,top=1.5cm,right=1cm,bottom=1cm]{geometry} % Document margins

\newcommand{\tab}[1]{\hspace{.2667\textwidth}\rlap{#1}}
\newcommand{\itab}[1]{\hspace{0em}\rlap{#1}}
\name{Jorge Antonio Garc\'{i}a Galicia} % Your name
%\address{1154 Morse Ave. Apt 108 \\ Sunnyvale, CA 94089} % Your address
%\address{\href{http://www.linkedin.com/in/jorgegarciagalicia}{http://www.linkedin.com/in/jorgegarciagalicia}} % Your secondary addess (optional)
\address{Sunnyvale, CA 94089 \\ \href{mailto:jorgeantonio49@gmail.com}{jorgeantonio49@gmail.com}} % A little bit more private for the abou.me page
%\address{(765) 543 - 9834 \\ \href{mailto:jorgeantonio49@gmail.com}{jorgeantonio49@gmail.com}} % Your phone number and email
\address{\href{http://www.linkedin.com/in/jorgegarciagalicia}{http://www.linkedin.com/in/jorgegarciagalicia}}

\usepackage{fancyhdr}
\showboxdepth=5
\showboxbreadth=5

\pagestyle{fancy}
\lhead{}
\chead{}
\rhead{Garc\'{i}a, Jorge pg.\thepage}
\cfoot{} % get rid of the page number 
\renewcommand{\headrulewidth}{0pt}
\renewcommand{\footrulewidth}{0pt}

\begin{document}
\thispagestyle{empty}
%----------------------------------------------------------------------------------------
%	EDUCATION SECTION
%----------------------------------------------------------------------------------------

\begin{rSection}{Education}

{\bf \href{http://www.purdue.edu}{PURDUE UNIVERSITY}} \hfill {\em West Lafayette IN} 
\\ \href{http://polytechnic.purdue.edu/degrees/phd-technology}{{\bf PhD} in Technology} \hfill {\em August 2017}
\\ \href{http://polytechnic.purdue.edu}{Purdue Polytechnic Institute}, \href{http://polytechnic.purdue.edu/departments/computer-graphics-technology}{Computer Graphics Department} \hfill {\em GPA: 3.79/4.0}
\\ Research areas: Additive Manufacturing and Computer Graphics 

{\bf \href{http://www.unam.mx}{NATIONAL AUTONOMOUS UNIVERSITY OF MEXICO}} \hfill {\em M\'{e}xico City} 
\\ \href{http://www.mcc.unam.mx}{{\bf Master} of Science in Computer Science} \hfill {\em September 2011}
\\ \href{https://www.iimas.unam.mx}{Institute of Applied Mathematics and Systems} \hfill {\em GPA: 9.4/10.0}
\\ Areas of study: Digital Image Processing and Computer Graphics 

\href{http://www.mac.acatlan.unam.mx}{{\bf Bachelor} of Science in Applied Mathematics and Computer Science} \hfill {\em May 2008}
\\ \href{https://www.acatlan.unam.mx}{School of Higher Studies Acatl\'{a}n} \hfill {\em GPA: 8.6/10.0}


\end{rSection}
%----------------------------------------------------------------------------------------
%	TECHNICAL STRENGTHS SECTION
%----------------------------------------------------------------------------------------

\begin{rSection}{Computer Skills}

\begin{tabular}{ @{} >{\bfseries}l @{\hspace{2ex}} l }
Programming Languages &  \textbf{Adv:} C/C++, GLSL. \textbf{Med:} Matlab, JavaScript \textbf{Beg:} Python, R\\
APIs / Frameworks & OpenGL, Vulkan, Qt, CUDA\\
Tools &  SVN, Git, \LaTeX, bash, phpESP, Wordpress\\
Software & Visual Studio, Eclipse, Gimp, Inkscape, Unity
\end{tabular}

\end{rSection}

%----------------------------------------------------------------------------------------
%	WORK EXPERIENCE SECTION
%----------------------------------------------------------------------------------------

\begin{rSection}{Experience}

\begin{rSubsection}{\href{http://about.google/}{Google LLC}}{Mountain View, CA}{Software Engineer, Platforms \& Ecosystems (prev Technical Solutions Engineer, Stadia)}{Mar 21 - present}
\item Shaping the \href{https://arvr.google.com/}{vision} on virtual and augmented reality. Unity and general Android development.
\item Supported partners in bringing their games to \href{http://stadia.google.com/}{Stadia}. DX11, DX12 and Vulkan.
\end{rSubsection}

\begin{rSubsection}{\href{http://www.nvidia.com}{Nvidia Corporation}}{Santa Clara, CA}{Senior Software Engineer, 3D graphics mobile (formerly browsers) team}{Aug 2017 - Feb 2021}
\item Contributed in making safety compliant a driver library (MISRA and CERT C) by following PLC
\item Developed several web based widgets for a robotics framework (ISAAC SDK)
\item Performed a comparison between ARCore and ARKit by building corresponding apps using Unreal Engine
\end{rSubsection}

\begin{rSubsection}{\href{http://www.adobe.com/}{Adobe Systems Incorporated}}{San Francisco, CA}{Research Intern, \href{http://research.adobe.com/}{Procedural Image Group}}{May 2016 - Aug 2016}
\item Designed a real time deferred rendering engine for an interactive sculpting application
\item Implemented shadow mapping, global ambient occlusion and PBR shading algorithms using GLSL shaders
\end{rSubsection}

\begin{rSubsection}{\href{http://www.nvidia.com}{Nvidia Corporation}}{Santa Clara, CA}{Software Developer Intern, OpenGL Driver, DirectX Driver}{May 2014 - Aug 2014, May 2015 - Aug 2015}
\item Ported OpenGL extensions to expos them in the driver API
\item Communicated with engineers from different companies around the world to delimit bug reports
\item Created an image format converter using nVidia assemble language
\item Wrote several shaders in HLSL for a video format conversion program
%\item Developed C functions to test algorithms inside the driver
\end{rSubsection}

%\begin{rSubsection}{National Autonomous University of Mexico}{Mexico City, Mexico}{Laboratory assistant, Software Development and Information Technology Unit}{Oct 2006 - Dec 2007}
%\item Built a web site using Linux, Apache, MySQL and PHP
%\item Developed various small applications in PHP and Java for data collection
%\item Administrated an SVN based control version system for the rest of the developers
%\end{rSubsection}

%\begin{rSubsection}{Computer Systems Technology}{Mexico City, Mexico}{Intern, Q.A. and developer}{Jan 2005 - Dec 2005}
%\item Developed various applications in Java for text processing
%\item Helped in a migration of a DB from Interbase to Oracle
%\item Worked migrating an application from PowerHouse to Java
%\end{rSubsection}

\end{rSection}

\clearpage



\begin{rSection}{Academic and Research Experience}

	\begin{rSubsection}{\href{http://www.purdue.edu}{Purdue University}}{West Lafayette, IN}{Teaching Assistant, \href{http://polytechnic.purdue.edu/departments/computer-graphics-technology}{Computer Graphics Department}}{Aug 2015 - May 2016, Aug 2016 - May 2017}
	\item Created didactic materials for \href{https://polytechnic.purdue.edu/sites/default/files/CGT-fall-2017.pdf}{CGT215} programming class
	\item Supervised students during the lab section of programming class
	\item Managed and graded homeworks for more than 50 students using Blackborad
	\end{rSubsection}

	\begin{rSubsection}{\href{http://www.purdue.edu}{Purdue University}}{West Lafayette, IN}{Research Assistant, \href{http://hpcg.purdue.edu/}{HPCG laboratory}}{Aug 2012 - May 2014, Aug 2014 - May 2015}
	\item Created a 3D visualization of the internal microstructures of batteries using OpenGL and CUDA
	\item Developed an algorithm for 3D bin packing optimization
	\item Contributed in developing software for analysis of road networks using Graph Theory techniques
	\end{rSubsection}
	
	\begin{rSubsection}{\href{http://www.unam.mx}{National Autonomous University of Mexico}}{Mexico City, Mexico}{Research Assistant, \href{https://turing.iimas.unam.mx/}{Institute of Applied Mathematics and Systems}}{Oct 2011 - Jun 2012}
	\item Developed programs for analysis of digital images for 3D reconstruction of blood vessels
	\item Operated a fondus camera to capture images of the retina
	\item Created a pipeline of several program using Bash to automatize the digital image processing
	\end{rSubsection}
	
	\begin{rSubsection}{\href{http://www.unam.mx}{National Autonomous University of Mexico}}{Mexico City, Mexico}{Lecturer, \href{https://www.acatlan.unam.mx}{School of Higher Studies Acatl\'{a}n}}{Aug 2008 - May 2012}
	\item Taught two classes: \href{https://www.acatlan.unam.mx/files/PlanesDeEstudio/MAC/4/Teoria_de_Graficas.pdf}{Graph Theory} and \href{https://www.acatlan.unam.mx/files/PlanesDeEstudio/MAC/7/Graficacion_por_Computadora.pdf}{Computer Graphics}
	\item Created didactic material including slides, quizzes and notes
	\item Provided feedback to students during extracurricular mentorship hours
	\end{rSubsection}
	
	\begin{rSubsection}{\href{http://www.unam.mx}{National Autonomous University of Mexico}}{Mexico City, Mexico}{\href{http://www.fciencias.unam.mx/directorio/63922}{Teaching Assistant}, \href{http://www.fciencias.unam.mx/}{School of Sciences}}{Jan 2009 - Dec 2010}
	\item Collaborated in teaching two classes: \href{http://www.fciencias.unam.mx/licenciatura/asignaturas/217/249}{Modern Geometry} and \href{http://www.fciencias.unam.mx/licenciatura/asignaturas/2017/1236}{Introduction to Computer Science}
	\item Created didactic material including demo programs and slides
	\item Guided students for installing software and learn best practices in programming
	\end{rSubsection}

\end{rSection}

\begin{rSection}{\href{http://scholar.google.com/citations?hl=en&user=p2dAy1MAAAAJ}{Publications}} \itemsep 1pt
\item \textit{\href{http://www.sciencedirect.com/science/article/pii/S2214860417302555}{Improving printing orientation for Fused Deposition Modeling printers by analyzing connected components}}, {\bf Jorge A. García Galicia} and Bedrich Benes. Additive Manufacturing, July 2018.
\item \textit{\href{http://dl.acm.org/doi/abs/10.1145/2948628.2948635}{Learning Geometric Graph Grammars}}, Fiser, M., Benes, B., {\bf Garcia, Jorge.}, Abdul-Massih, M., Aliaga, D., and Krs, V. Proceedings of the 32Nd Spring Conference on Computer Graphics, 2016.
\item \textit{\href{http://dl.acm.org/doi/abs/10.1145/2897824.2925958}{Connected fermat spirals for layered fabrication}}, Zhao, H., Gu, F., Huang, Q., {\bf Garcia, J.}, Chen, Y., Tu, C., Benes, B., Zhang, H., Cohen-Or, D., Chen, B. ACM Transactions on Graphics (SIGGRAPH), July 2016.
\item \textit{\href{http://www.mitpressjournals.org/doi/abs/10.1162/LEON_a_01090}{Yturralde: impossible figure generator}}, Esteban Garc\'{i}a Bravo and {\bf Jorge A. Garc\'{i}a.} SIGGRAPH '15 ACM SIGGRAPH Art Papers.
\item \textit{\href{http://onlinelibrary.wiley.com/doi/abs/10.1111/cgf.12353}{PackMerger: A 3D Print Volume Optimizer}}, Vanek, J., {\bf Garcia, J.}, Benes, B., Mech, R., Carr, N., Stava, O., and Miller, G. Computer Graphics Forum, September 2014.
\item \textit{\href{http://onlinelibrary.wiley.com/doi/abs/10.1111/cgf.12437}{Clever Support: Efficient Support Structure Generation for Digital Fabrication}}, Vanek, J., {\bf Garcia, J.} and Benes, B. Computer Graphics Forum, August 2014.
\end{rSection}

% \begin{rSection}{Other nominations and awards} \itemsep -6pt
% \item Polytechnic Institute Summer Research Grant 2017.
% \item Scholarship for international graduate studies, National Council of Science and Technology
% \item Senator, The Purdue Graduate Student Senate
% \item Scholarship for graduate studies, National Council of Science and Technology
% \item Technical Council member, School of Higher Studies Acatl\'{a}n
% \end{rSection}

%----------------------------------------------------------------------------------------
%\begin{rSection}{Relevant Courses}
%\\ \itab{Computational Geomtery } \tab{}  \tab{Scientific visualization}
%\\ \itab{Multispectral image analysis } \tab{}  \tab{Time series} 
%\\ \itab{Metaehuristics for combinatorial optimization } \tab{}  \tab{Stocastich process} 
%\\ \itab{Digital fabrication } \tab{} \tab{Complex variable }
%\end{rSection}

\end{document}
