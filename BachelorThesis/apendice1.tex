\chapter{Sobre el software libre}
Un objetivo secundario cuando empecé a hacer esta tesis, fue que toda ella fuera hecha con Software Libre. Al final tengo que admitir que esto no fue llevado a cabo por completo, pues todas las figuras del primer capítulo con la excepción de la~\ref{OsciAmor:fig} fueron hechas en AutoCAD (http://www.autodesk.es).

Aun así, pienso que este objetivo se cumplió parcialmente, pues fuera de lo antes mencionado todo se hizo en Software Libre; el desarrollo del programa fue hecha sobre GNU/Linux en particular sobre la distribución Ubuntu (www.ubuntu.com).

Para programar se utilizó gcc (http://gcc.gnu.org/), junto con Mesa (http://www.mesa3d.org/) y freeglut (http://freeglut.sourceforge.net/), la biblioteca glui (http://www.cs.unc.edu/rademach/glui/) también es software libre. Como IDE utilice Geany (http://geany.uvena.de/) y debo decir que estoy muy satisfecho con él.

Cuando hubo necesidad de hacer pruebas en Windows, también se ocuparon programas libres: se compiló con Dev-C++ (http://www.bloodshed.net/devcpp.html) como IDE y con Migwn (http://www.mingw.org/) como compilador junto con glui y freeglut.

El texto de la tesis se escribió en \LaTeX, en Ubuntu utilice tetex (http://web.bilkent.edu.tr/History/valley/tetex-index.html)y en Windows miktex (http://miktex.org/).

También fue de muchísima ayuda contar con un repositorio para guardar los avances del proyecto, esto me dio la enorme ventaja de poder trabajar en cualquier computadora que tuviera acceso a internet. Para eso se hizo uso de apache (http://www.apache.org/) y de subversion (http://subversion.tigris.org/).
