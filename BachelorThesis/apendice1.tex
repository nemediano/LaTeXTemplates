\chapter*{Apéndice: Sobre el software libre}
\addcontentsline{toc}{chapter}{Apéndice}
\markboth{APENDICE}{APENDICE}
Un objetivo secundario cuando empecé a hacer esta tesis, fue que toda ella fuera hecha con Software Libre. Al final tengo que admitir que esto no fue llevado a cabo por completo, pues todas las figuras del primer capítulo con la excepción de la~\ref{OsciAmor:fig} fueron hechas en \href{http://www.autodesk.es}{AutoCAD}.

Aun así, pienso que este objetivo se cumplió parcialmente, pues fuera de lo antes mencionado todo se hizo en Software Libre; el desarrollo del programa fue hecha sobre GNU/Linux en particular sobre la distribución \href{www.ubuntu.com}{Ubuntu}.

Para programar se utilizó \href{http://gcc.gnu.org/}{gcc}, junto con \href{http://www.mesa3d.org/}{Mesa} y \href{http://freeglut.sourceforge.net/}{freeglut}, la biblioteca \href{http://www.cs.unc.edu/rademach/glui/}{glui} también es software libre. Como IDE utilice \href{http://geany.uvena.de/}{Geany} y debo decir que estoy muy satisfecho con él.

Cuando hubo necesidad de hacer pruebas en Windows, también se ocuparon programas libres: se compiló con \href{http://www.bloodshed.net/devcpp.html}{Dev-C++} como IDE y con \href{http://www.mingw.org/}{Migwn} como compilador junto con glui y freeglut.

El texto de la tesis se escribió en \LaTeX, en Ubuntu utilice \href{http://web.bilkent.edu.tr/History/valley/tetex-index.html}{tetex} y en Windows \href{http://miktex.org/}{miktex}.

También fue de muchísima ayuda contar con un repositorio para guardar los avances del proyecto, esto me dio la enorme ventaja de poder trabajar en cualquier computadora que tuviera acceso a internet. Para eso se hizo uso de \href{http://www.apache.org/}{apache} y de \href{http://subversion.tigris.org/}{subversion}.
