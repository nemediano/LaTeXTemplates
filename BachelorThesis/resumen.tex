\chapter*{Resumen}
Este trabajo trata de como crear una simulación gráfica de un cuerpo flexible.
El cuerpo flexible, esta construido --como es costumbre en graficación por computadora-- por una malla tridimensional. Ademas, en éste trabajo; se asume que la aristas de la malla son resortes y que el cuerpo flexible esta relleno de aire.

Para hacer una animación basada en Física, se definen un conjunto de fuerzas que actuan sobre los vértices de la malla. Después, se usa la fuerza acumulada para integrar la ecuación de Newton y obtener las velocidades y posiciones de los vértices en la escena.

Las fuerzas aplicadas son: la gravedad, los resortes-amortiguadores y la fuerza de presión. Esta última resultado de usar la ecuación del gas ideal.

Se describe a detalle, la creación de un programa que implementa todas estas ideas para hacer una animación interactiva. La implementación se hizo en lenguaje C++ y se usa OpenGL como biblioteca para desplegar graficos.
