\chapter*{Resumen}
Este trabajo trata de como visualizar campos escalares que han pasado por un proceso de digitalización.
En particular se usa una técnica de graficación por computadora conocida como visualización por superficies o \emph{surface rendering}.

Se asume que se tienen conjuntos digitales de datos que provienen de haber muestreado de manera uniforme el espacio en tres dimensiones.
Asumimos que este muestreo está hecho en una rejilla rectangular y por lo tanto tenemos una imagen digital en 3D o volumen.
Hacemos dos suposiciones importantes sobre el volumen.
Primero, que es una buena aproximación del campo escalar y segundo que no tenemos información de la manera como se realizó la digitalización.
