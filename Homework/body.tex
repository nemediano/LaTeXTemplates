A researcher interested in traffic safety and alcohol abuse was interested in study the relationship between the amount of alcohols consumed and driver reaction times. To investigate this issue the researcher recruited 8 college students to participate in the experiment were reaction time of the participant was measured under four conditions. The first condition involved no consumption of alcohol. The second condition involves the consumption of two ounces of alcohol 30 minutes prior to the reaction time measure. The third condition involved the consumption of 4 ounces of alcohol 30 minutes prior to the reaction time measure. And condition four involved the consumption of 6 ounces of alcohol 30 minutes previous to the reaction time measure. The data were collected oven an 8 week period and the order in which the subject received the alcohol condition was counter balanced. The results are reported below. The scores are in seconds. Based on these results what condition could you draw about the relationship between the alcohol consumption and reaction time.

\begin{table}[!htb]
    %\caption{Reading Power Test Scores}
		\centering
        \begin{tabular}{crrrr}
						 & \multicolumn{4}{c}{Alcohol consumption} \\
						\cline{2-5}
            Student & None & 2 oz & 4 oz & 6 oz \\
						1 & 3 & 4 & 7 & 7 \\
						2 & 6 & 5 & 8 & 8 \\
						3 & 3 & 4 & 7 & 9 \\
						4 & 3 & 3 & 6 & 8 \\
						5 & 1 & 2 & 5 & 10 \\
						6 & 2 & 3 & 6 & 10 \\
						7 & 2 & 4 & 5 & 9 \\
						8 & 2 & 3 & 6 & 11 \\
        \end{tabular}
\end{table}

\begin{enumerate}
	\item Assuming the average correlation between the levels of alcohol equals 0.5 and a small effect is important to detect, how many subject are needed for this study if $\alpha = 0.05$ and $\beta = 0.20$.
	
  \item Is there any assumption to indicate that the assumption of sphericity has been violated? state an appropriate test statistic, degrees of freedom, and $p$-value also provide a descriptive statistic to support your answer.
	
	\item Is there any evidence to indicate that the reaction time is affected by the consumption of alcohol? State the test statistic, degrees of freedom and $p$-value for the appropriate test of alcohol main effect. 
	
	\item State the strength of the relationship between alcohol and reaction time.
	
	\item Assume that only pairwise contrast with no alcohol are of interest. Use the Boferroni method to control the experiment wise error rate equal to 0.05 and interpret the result of the analysis.
	
\end{enumerate}
  
%\begin{table}[!htb]
	%\centering
	%\caption{Attitude}
	%\centering
		%\begin{tabular}{lrrrrr}
			%\toprule
			%Source & $df$  & Sum of square & Mean Square & $F$  & $p$-value \\
			%\midrule
			%Between $_A$ & 3  & 390.10 & 130.03 & 13.26 & $< 0.000$ \\
			%Within  $_{S / A}$ & 36 & 353.00 & 9.81 &  & \\
			%\cline{1-3}
			%Total   & 39 & 743.10 &  & &  \\
			%\bottomrule
		%\end{tabular}
%\end{table}		
