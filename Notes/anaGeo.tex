\section{Analytic Geometry}

I will use the convention from~\cite{Buss2003}.
We use lower case names in normal font to represent scalars: $s$, $t$, $v$.
Bolt font with lower case names to represent vectors: $\mathbf{a}$, $\mathbf{b}$, $\mathbf{p}$.
Upper case names in bold face to represent set of points in the space: $\mathbf{L}$, $\mathbf{R}$ (like lines or rays).
Finally, upper case names in normal font to represent matrices: $P$, $V$, $M$.

The vectors are column vectors. In other words:
$$ \mathbf{v} = \begin{pmatrix}
  x \\ 
  y \\
  z
 \end{pmatrix} $$

Then we can use syntax like: $A \mathbf{x} = \mathbf{b}$.
To make the reading easier, in case I will need to specify a vector inside a paragraph I will use $\mathbf{v} = \langle x, y, z \rangle$

In the notation, we do not make any distinction between vectors and points, therefore the text always give context to clarify this. 

\subsection{Line, ray and segment}

In most CG applications, we use two \emph{different} points; let's call them $\mathbf{a}$ and $\mathbf{b}$ (with $\mathbf{a} \neq \mathbf{b}$) to specify a line: $\mathbf{L}$. 

However, in analytic geometry we prefer a parametric equation for the line: $\mathbf{L}(t)$.
We use a point $\mathbf{p}$, a directional vector $\mathbf{v}$ and a scalar $t$ as parameter : $\mathbf{L}(t) = \mathbf{p} + t \mathbf{v}$, with $t \in (-\infty, \infty)$.

To transform between this two representations we do: $\mathbf{v} = \mathbf{b} -\mathbf{a}$. And, then the line is: 

\begin{equation}
\mathbf{L}(s) = \mathbf{a} + s (\mathbf{b} -\mathbf{a}) = \mathbf{a} + s \mathbf{v} 
\label{eq:line}
\end{equation}
, with $s \in (-\infty, \infty)$. 

Note that the order of $\mathbf{a}$ and $\mathbf{b}$ is relevent, because of the way we define $\mathbf{v}$: it points from $\mathbf{a}$ to $\mathbf{b}$.
Also note that $\mathbf{v}$ is not necesarlly a unit vector. 

\begin{figure}[htb]
  \centering
  \includegraphics[width=0.85\textwidth]{img/line}
  \caption{Line identified by two points. See Equation~\ref{eq:line} }
  \label{fig:line}
\end{figure}

The Figure~\ref{fig:line} shows the line from Equation~\ref{eq:line} indentified by the points $\mathbf{a}$ and $\mathbf{b}$.
On these conditions; we can give continuos values to the parameter $s$ in the Equation~\ref{eq:line}, to obtain points on the line $\mathbf{L}$.
Particulary, if we let $s = 0$ we get the point $\mathbf{a}$ and if we let $s = 1$ we get  
$\mathbf{b}$.
Moreover, any value $0 \leq s \leq 1$ will get us a point $\mathbf{p}$ inside the line segment $\overline{\mathbf{a} \mathbf{b}}$.
Any value $s > 1$ will get a $\mathbf{p} \in \mathbf{L}$, $\mathbf{p} \notin \overline{\mathbf{a} \mathbf{b}}$ that is closer to $\mathbf{b}$ than it is to to $\mathbf{a}$.
Conversely, any $s < 0$ will get a $\mathbf{p} \in \mathbf{L}$, $\mathbf{p} \notin \overline{\mathbf{a} \mathbf{b}}$ that is closer to $\mathbf{a}$ than it is to $\mathbf{b}$.

We can use this last observations to identify segments and rays.
A \emph{ray} from $\mathbf{a}$ in the direction of $\mathbf{b}$ is the one that has Equation~\ref{eq:line}, with the constrain that the parameter must be $s \geq 0$.
The line \emph{segment} from $\mathbf{a}$ to $\mathbf{b}$, denoted by $\overline{\mathbf{a} \mathbf{b}}$, is the one that also follows Equation~\ref{eq:line}, with the constrain that   $0 \leq s \leq 1$.

\subsection{Point to line distance}

Given a line $\mathbf{L}$ and a point $\mathbf{p}$, we want to identify the distant $d$ from $\mathbf{p}$ to $\mathbf{L}$.
It's clear that if $\mathbf{p} \in \mathbf{L}$, then $d = 0$.
Now, if $\mathbf{p} \notin \mathbf{L}$, we want the distant between $\mathbf{p}$ and a certain point $\mathbf{q}$, such that $\mathbf{q} \in \mathbf{L}$ and $d(\mathbf{q}, \mathbf{p})$ is the minimal to all other points in $\mathbf{L}$.

\begin{figure}[htb]
  \centering
  \includegraphics[width=0.85\textwidth]{img/line}
  \caption{Distance $d(\mathbf{p}, \mathbf{L})$ between a point $\mathbf{p}$ and a line $\mathbf{L}$}
  \label{fig:point2line}
\end{figure}

In other words we want the minimal distance from $\mathbf{p}$ to $\mathbf{L}$.
The Figure~\ref{fig:point2line} depicts this situation.
If we start sliding a point on $\mathbf{L}$ (strating lets say at $\mathbf{p}$) and evaluating the distance beween that point and $\mathbf{p}$, we can see that the distance start getting smaller as we come close to $\mathbf{p}$, get his minimal value when we are right bellow $\mathbf{q}$ and then increases again. 
Indeed, the distance we are looking is between $\mathbf{p}$ and the point $\mathbf{q}$ that is in the intesection between the lines $\mathbf{L}$ and $\mathbf{M}$
Where $\mathbf{M}$ is the line perpendicular to $\mathbf{L}$ that toches $\mathbf{p}$. 

Before proceding I put a quick reminder of the formula to calculate the projection of a vector $\mathbf{x}$ into another vector $\mathbf{y}$.

\begin{equation}
\proj_{\mathbf{y}}\mathbf{x} = \dfrac{(\mathbf{x} \cdot \mathbf{y})}{|\mathbf{y}|} \dfrac{\mathbf{y}}{|\mathbf{y}|} = \dfrac{(\mathbf{x} \cdot \mathbf{y})}{|\mathbf{y}|^2}\mathbf{y}
\label{eq:proj}
\end{equation}

In order to calculate the distance we use the Equation~\ref{eq:proj} to find the corresponding parametric value $s_q$ for $\mathbf{q}$ on $\mathbf{L}$: 
\begin{equation}
s_q = \dfrac{(\mathbf{p} - \mathbf{a}) \cdot (\mathbf{b} - \mathbf{a})}{|\mathbf{b} - \mathbf{a}|^2}
\label{eq:scalarq}
\end{equation}
Then the point is:
\begin{equation}
\mathbf{q} = \mathbf{a} + s_q (\mathbf{b} -\mathbf{a})
\label{eq:pointq}
\end{equation}
And finally, the distance is $d(\mathbf{p}, \mathbf{L}) = d(\mathbf{p}, \mathbf{q})$.
\subsubsection{Point to segment distance}

Another common derived problem is to find the distance of a point $\mathbf{p}$ to a segment $\overline{\mathbf{a}\mathbf{b}}$.
See Figure~\ref{fig:point2line}, we see the distance of the different points depend on how are they placed with respect of the segment.
For the point $\mathbf{p}_1$ the distance to the segment is the distance $d(\mathbf{p}_1, \mathbf{a})$, for the point $\mathbf{p}_2$ the distance is $d(\mathbf{p}_2, \mathbf{b})$ and for the point $\mathbf{p}_3$ the distance is actually the distance $d(\mathbf{p}_3, \mathbf{L})$ where $\mathbf{L}$ is the line defined by $\mathbf{a}$ and $\mathbf{b}$.

The observations at the end of Section and Section give us a nice Algorithm to find this distance.

{\centering
\begin{minipage}{\linewidth}
  \begin{algorithm}[H]
    \caption{Distance between a point $\mathbf{p}$ and a line segment $\overline{\mathbf{a}\mathbf{b})}$}
    \label{alg:euclid}
    \begin{algorithmic}[1] % The number tells where the line numbering should start 0 for no number
      \Procedure{DistancePoint2Segment}{$\mathbf{a},\mathbf{b},\mathbf{p}$} \Comment{$\mathbf{a}$ and $\mathbf{b}$ are vectors and $\mathbf{p}$ is a point all of the same dimension}
        \If{$\mathbf{a} = \mathbf{b}$} \Comment{There is no line it's just a point}
          \State \textbf{return} $d(\mathbf{a},\mathbf{p})$
        \EndIf
        \State $s \gets \frac{(\mathbf{p} - \mathbf{a}) \cdot (\mathbf{b} - \mathbf{a})}{|\mathbf{b} - \mathbf{a}|^2}$ \Comment{Use Equation~\ref{eq:scalarq} to get parametric value of $\mathbf{q}$}
        \If{$s < 0$} \Comment{$\mathbf{p}$ is closer to $\mathbf{a}$ than to $\overline{\mathbf{a}\mathbf{b}}$}
          \State $r \gets d(\mathbf{a},\mathbf{p})$
         \ElsIf{$s > 1$} \Comment{$\mathbf{p}$ is closer to $\mathbf{b}$ than to $\overline{\mathbf{a}\mathbf{b}}$}
          \State $r \gets d(\mathbf{b},\mathbf{p})$
        \Else \Comment{$\mathbf{p}$ is over $\overline{\mathbf{a}\mathbf{b}}$}
          \State $\mathbf{q} \gets \mathbf{a} + s (\mathbf{b} - \mathbf{a})$ \Comment{Calculate $\mathbf{q}$}
          \State $r \gets d(\mathbf{q},\mathbf{p})$
        \EndIf
        \State \textbf{return} $r$ \Comment{Distance is stored in $r$}
      \EndProcedure
    \end{algorithmic}
  \end{algorithm}
\end{minipage}
\par
}

\subsection{Ray to segment intersection}

Another common problem is to test if a ray $\mathbf{R}$ intersects a line segment $\mathbf{S}$.
The ray its defined by a point of origin $\mathbf{o}$ and a directional vector $\mathbf{u}$.
As usual, the segment is defined by his two end points $\mathbf{a}$ and $\mathbf{b}$.

In the scenario described above, the ray is $\mathbf{R}(s) = \mathbf{o} + s \mathbf{v}$ with $s \geq 1$ and the segment is $\mathbf{S}(t) = \mathbf{a} + t (\mathbf{b} - \mathbf{a})$ with $0 \leq t \leq 1$.
If $\mathbf{R}(s)$ intersects $\overline{\mathbf{a}\mathbf{b}}$, then $\mathbf{R}(s) = \mathbf{S}(t)$ for some values of $\mathbf{s}$ and $\mathbf{t}$.

\begin{align*}
\mathbf{a} + t (\mathbf{b} - \mathbf{a}) &= \mathbf{o} + s \mathbf{v} \\
\mathbf{a} - \mathbf{o} &= s \mathbf{v} + t (\mathbf{a} - \mathbf{b})
\end{align*}

\subsection{Belonging test on simple polygons}

