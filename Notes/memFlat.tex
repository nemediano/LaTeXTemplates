\section{Memory flatenig}

A common design problem in multidimensional data situations is memory layout representation. Internally, computers always represent memory in a linear fashion, the idea of \emph{mutidimensional array} it's just sugar syntax added by high level programming lenguajes.

\subsection{One dimensional array}

Given said that, sometimes allocations for such constructs can become dificult to use or declare by themselves. Especially in C++. Lets see an example:

{\centering
\begin{minipage}{\linewidth}
  \begin{listing}[H]
  \inputminted[
  xleftmargin=1.5cm,  %without this option line number goes wrong
  %frame=lines,
  framesep=0.5cm,
  baselinestretch=1.2,
  fontsize=\footnotesize,
  linenos,
  firstline=15, %If you omit this two fields, the whole file is pulled
  lastline=21
  ]{cpp}{src/ArrayDimensions.cpp}
  \caption{One dimensioal array (by means of \mintinline{cpp}{std::vector} class) used}
  \label{lst:1Dexample}
  \end{listing}
\end{minipage}
\par
}
\vspace{0.5cm}
So far, everything looks good, the sintax is clear, braket operator is helping us.
Conceptually, we have a series of variables with the same type that live in a contiguos space in memory. See Figure~\ref{fig:1D}.

\begin{figure}[htp]
  \centering
  \begin{subfigure}[b]{0.35\textwidth}
    \includegraphics[width=\textwidth]{img/array1D}
    \caption{Logical memory layout representation.}
  \label{fig:1a}
  \end{subfigure}
  \hspace*{4cm}
  \begin{subfigure}[b]{0.25\textwidth}
    \includegraphics[width=\textwidth]{img/arrow1D}
    \caption{Dimensions represented.}
    \label{fig:1b}
  \end{subfigure}
  \caption{One dimensional array representation.}
  \label{fig:1D}
\end{figure}

\subsection{Two dimensional data}

The problem becomes evident as data start to include more dimensions.
In a bidimensional array we have data that has two dimensions.
One can think about it, like the data it's stored in a table, the situation is the one depicted in Figure~\ref{fig:2a}.
Now, remember this is just an abstraction; provided by our programming language: C++, in this case, since memory is actually \emph{close} to linear in a computer.

\begin{figure}[htp]
  \centering
  \begin{subfigure}[b]{0.35\textwidth}
    \includegraphics[width=\textwidth]{img/array2D}
    \caption{Logical memory layout representation.}
  \label{fig:2a}
  \end{subfigure}
  \hspace*{4cm}
  \begin{subfigure}[b]{0.25\textwidth}
    \includegraphics[width=\textwidth]{img/arrow2D}
    \caption{Dimensions represented.}
    \label{fig:2b}
  \end{subfigure}
  \caption{Two dimensional array representation.}
  \label{fig:2D}
\end{figure}

Usually, the C++ way of modeling this is creating an array, such that each element of it, is also an array. See the code in Listing~\ref{lst:2dexample}

{\centering
\begin{minipage}{\linewidth}
  \begin{listing}[H]
  \inputminted[
  xleftmargin=1.5cm,  %without this option line number goes wrong
  %frame=lines,
  framesep=0.5cm,
  baselinestretch=1.2,
  fontsize=\footnotesize,
  linenos,
  firstline=23, %If you omit this two fields, the whole file is pulled
  lastline=38
  ]{cpp}{src/ArrayDimensions.cpp}
  \caption{Two dimensioal array example}
  \label{lst:2dexample}
  \end{listing}
\end{minipage}
\par
}
\vspace{0.5cm}
Now, there is three main points to highlight in the sample:
\begin{enumerate}

\item The type of the array has change.
      Now is \emph{a vector of int vectors}.
      This led us to have a longer initialization for it too.
      Imagine that you create functions to receive or return such arrays.
      The sintax to declare the parameters will become difficult.

\item We still have the advantage of a very elegant sintax to use the array: \mintinline{cpp}{b[j][i]} give us the element on column $i$ of row $j$.

\item The sintax is backwads with respect to mathematic notation.
      See Figure~\ref{fig:2D} again.
      The index \mintinline{cpp}{i} is transversin along the $x$ dimension and the index \mintinline{cpp}{j} along the $y$ dimension.
      Therefore, the element \mintinline{cpp}{b[j][i]} it's analogous to $b(x, y)$ in math.
 \end{enumerate}

\subsection{Memory flatenig explained}

Here is when the technique called \emph{memory flattening} becomes useful.
In simple words: This is to mimic what the compiler is doing for us: use linear data.
But at the same time you also provide an abstraction, so you can access it in a similar fashion that if your data had more dimensions.
In other works you simulate the access via indices for all the required dimensions.

See the Figure~\ref{fig:Flat}.
The data is equivalent to the one shown in Figure~\ref{fig:2D} in the sense that represents a two dimensional table, whose first dimension $x$ (number of columns: $5$), is indexed by $i \in \{0, 1, 2, 3, 4\}$ and whose second dimension $y$ (number of rows: $4$) is indexed by $j \in \{0, 1, 2, 3\}$.
However, the data is stored in a one dimensional array of size $5 \cdot 4 = 20$.
The actual indices of the array $ \in \{ 0, 1, \ldots, 19 \}$ and are shown on the bottom of of Figure~\ref{fig:Flat}.
However, we also provide an abstraction to access this data via simulated indices \mintinline{cpp}{[j][i]} shown in the top of Figure~\ref{fig:Flat}

\begin{figure}[htb]
  \centering
  \includegraphics[width=0.85\textwidth]{img/arrayFlat}
  \caption{2D array represented as a \emph{flattened} array. This represents the same data as Figure~\ref{fig:2D}. Order of the indices is \mintinline{cpp}{[j][i]}}
  \label{fig:Flat}
\end{figure}

This has the advantage that all the data is store in a one dimensional array.
Therefore, you \emph{ensure} all your data is actually contigous in memory.
And no matter how many dimensions your data has, the type is always a one dimensional array.

Also, provided that we acces the data in the \emph{correct way}, the access tends to be faster than the alternative shown in Listing~\ref{lst:2dexample}.
The \emph{correct way} means that the inner loop moves along the $x$ dimension and the outer loop transverses the second dimension $y$.
The reason for this improvement becomes evident if we see Figure~\ref{fig:Flat} again: in this way, we are actually accesing a linear array in his natural linear order.

In order to implement this technique we will need two mappings:
First, a function $g:\mathbb{Z}^n \rightarrow \mathbb{Z}$, to map the indices in $n$ dimensions to the one dimensional array.
And second, the inverse mapping $h:\mathbb{Z} \rightarrow \mathbb{Z}^n$ to retrive the indices back given a position in the data buffer.

The first function $g$ it is actually the most useful to us, since it will provide with the interface of accesing our data.
See the Figure~\ref{fig:2D}.
Assume that our data hast two dimensions ($x$ and $y$ --or \emph{width} and \emph{height}-- if you like).
That we use two indices $i$ and $j$ to transverse it.
And assume also that the number of cells along those dimensions are \mintinline{cpp}{WIDTH} and \mintinline{cpp}{HEIGHT} and that the indices are zero based.
In Listing~\ref{lst:functions2D} we show the implementation of such function.

We can see that the transformations are actually very simple.
In fact, one of them is the integer division and the other one is the modulo.
Also note, that as long as we check for the boundaries of the values for both indices, only one of the sizes in this case \emph{witdh} is used in the actual calculation of the indices.

{\centering
\begin{minipage}{\linewidth}
  \begin{listing}[H]
  \inputminted[
  xleftmargin=1.5cm,  %without this option line number goes wrong
  %frame=lines,
  framesep=0.5cm,
  baselinestretch=1.2,
  %fontsize=\footnotesize,
  linenos,
  firstline=88, %If you omit this two fields, the whole file is pulled
  lastline=105
  ]{cpp}{src/FlatMemmory.cpp}
  \caption{Function $g$ to map real index into simulated 2D indices}
  \label{lst:functions2D}
  \end{listing}
\end{minipage}
\par
}
\vspace{0.5cm}

For completeness, we also show Listing~\ref{lst:functions2D2} that defines the inverse function $h$, to retrive the index in the one dimensional array from the two simulated indices in 2D.

{\centering
\begin{minipage}{\linewidth}
  \begin{listing}[H]
  \inputminted[
  xleftmargin=1.5cm,  %without this option line number goes wrong
  %frame=lines,
  framesep=0.5cm,
  baselinestretch=1.2,
  %fontsize=\footnotesize,
  linenos,
  firstline=107, %If you omit this two fields, the whole file is pulled
  lastline=121
  ]{cpp}{src/FlatMemmory.cpp}
  \caption{Function $h$ to map simulated 2D indices into the real index, this is the inverse of $g$}
  \label{lst:functions2D2}
  \end{listing}
\end{minipage}
\par
}
\vspace{0.5cm}

\subsection{Three dimensional data}

Now, lets look at the more complex --but still very common-- scenario of three dimensions. The situation is the one depicted in Figure~\ref{fig:3D}. We include a new dimension in the data: indexed by $k$ along the $z$ axis, we call it \emph{depth}. Note that even when the frame of reference in Figure~\ref{fig:3b}, looks weird; it is a \emph{right handed} frame of reference. 

\begin{figure}[htp]
  \centering
  \begin{subfigure}[b]{0.35\textwidth}
    \includegraphics[width=\textwidth]{img/array3D}
    \caption{Logical memory layout representation.}
  \label{fig:3a}
  \end{subfigure}
  \hspace*{4cm}
  \begin{subfigure}[b]{0.2\textwidth}
    \includegraphics[width=\textwidth]{img/arrow3D}
    \caption{Dimensions represented.}
    \label{fig:3b}
  \end{subfigure}
  \caption{Three dimensional array representation.}
  \label{fig:3D}
\end{figure}

In the normal mutidimensional array synatx of C++, this will become, an array of arrays of arrays of ints. See Listing~\ref{lst:3dexample}. We transvers it using three nested loops, where the most outer one is the one indexed by the new dimension.

{\centering
\begin{minipage}{\linewidth}
  \begin{listing}[H]
  \inputminted[
  xleftmargin=1.5cm,  %without this option line number goes wrong
  %frame=lines,
  framesep=0.5cm,
  baselinestretch=1.2,
  fontsize=\footnotesize,
  linenos,
  firstline=40, %If you omit this two fields, the whole file is pulled
  lastline=64
  ]{cpp}{src/ArrayDimensions.cpp}
  \caption{Three dimensional array example in C++}
  \label{lst:3dexample}
  \end{listing}
\end{minipage}
\par
}
\vspace{0.5cm}

For memory flatening, we will need a new version of functions $g$ and $h$, that depends in a new extra parameter: \mintinline{cpp}{DEPTH}.
In Listing~\ref{lst:functions3D}, we can see the new function $g$.

{\centering
\begin{minipage}{\linewidth}
  \begin{listing}[H]
  \inputminted[
  xleftmargin=1.5cm,  %without this option line number goes wrong
  %frame=lines,
  framesep=0.5cm,
  baselinestretch=1.2,
  fontsize=\footnotesize,
  linenos,
  firstline=123, %If you omit this two fields, the whole file is pulled
  lastline=143
  ]{cpp}{src/FlatMemmory.cpp}
  \caption{New function $g$ to map real index into simulated 3D indices}
  \label{lst:functions3D}
  \end{listing}
\end{minipage}
\par
}
\vspace{0.5cm}

I will talk about the form of the function for the second result (the $y$ that represent the \mintinline{cpp}{j} index) in Section~\ref{sec:moreDim}. Finally, for completenees; the new function $h$ is presented in Listing~\ref{lst:functions3D2}

{\centering
\begin{minipage}{\linewidth}
  \begin{listing}[H]
  \inputminted[
  xleftmargin=1.5cm,  %without this option line number goes wrong
  %frame=lines,
  framesep=0.5cm,
  baselinestretch=1.2,
  fontsize=\footnotesize,
  linenos,
  firstline=145, %If you omit this two fields, the whole file is pulled
  lastline=163
  ]{cpp}{src/FlatMemmory.cpp}
  \caption{A new function $h$ to map simulated 3D indices into the real index, this is the inverse of $g$}
  \label{lst:functions3D2}
  \end{listing}
\end{minipage}
\par
}

\subsubsection{Generalizing to more dimensions}
\label{sec:moreDim}

In the case of two dimensions the calculation for function $g$ were simply the integer division and the modulo. In the case of three dimensions we also have a modulo and a division (Altrought, the parameters change in the later one). However, the interesting case is that of the middle value: it has both operations.

This is actually the general case. Indeed, the correct transformation to get the simulated index of a given dimension, is as follows:
\begin{enumerate}
  \item Divide the real index by the size of the unit of all the previous dimensions.
  \item Take the result of the previous operation and  calculate his modulo with the maximum size of the given dimension.
\end{enumerate}

Let me present in Listing~\ref{lst:functions3D3} an alternative to the function $g$ showed in Listing~\ref{lst:functions3D} using the general appproach.

{\centering
\begin{minipage}{\linewidth}
  \begin{listing}[H]
  \inputminted[
  xleftmargin=1.5cm,  %without this option line number goes wrong
  %frame=lines,
  framesep=0.5cm,
  baselinestretch=1.2,
  fontsize=\footnotesize,
  linenos,
  firstline=187, %If you omit this two fields, the whole file is pulled
  lastline=205
  ]{cpp}{src/FlatMemmory.cpp}
  \caption{Alternative $g$ function for 3D. Using the general form of the algorithm}
  \label{lst:functions3D3}
  \end{listing}
\end{minipage}
\par
}
\vspace{0.5cm}

Now, it becomes clear why in Listing~\ref{lst:functions3D} we did not need the full forms.
For the $x$ variable the previous dimension has size one, and we do not need to divide by one.
As for the last variable $z$, we know that \mintinline{cpp}{index} is striclly less than \mintinline{cpp}{WIDTH * HEIGHT * DEPTH}, and we divide it by \mintinline{cpp}{WIDTH * HEIGHT}, so it should be a number stricltly less than \mintinline{cpp}{DEPTH}, therefore we do not need to take the modulo.

Now that the the logic of the function $g$ is clear; we can also come up with a general rule for the inverse function $h$.

\begin{enumerate}
  \item For each dimension calculate a term. The term is the size of the unit in the previos dimension times the index in the current dimension.
  \item Accumulate (add) all the terms to calculate the real index.
\end{enumerate}

See Listing~\ref{lst:functions3D4} for the \emph{general rule} implementation.

{\centering
\begin{minipage}{\linewidth}
  \begin{listing}[H]
  \inputminted[
  xleftmargin=1.5cm,  %without this option line number goes wrong
  %frame=lines,
  framesep=0.5cm,
  baselinestretch=1.2,
  fontsize=\footnotesize,
  linenos,
  firstline=166, %If you omit this two fields, the whole file is pulled
  lastline=185
  ]{cpp}{src/FlatMemmory.cpp}
  \caption{Alternative to function $h$ using the general rule}
  \label{lst:functions3D4}
  \end{listing}
\end{minipage}
\par
}
