\section{Notes from book}

While I was reading the book~\cite{Gottschling2015}, I had the need to take notes at all the parts that were specially important either because I did not know them or because I am not always aware or their importance (i.e: I tent to forget them).

\subsection{Chapter 1}
\begin{itemize}
 \item Brace initialization cannot be narrowed (pp 10)
 \item Do not use macros (pp 10)
 \item Addresable is called an \emph{lvale} (pp 13)
 \item Mathemathic expresions should not have side effects. (pp 12)
 \item Always use \mintinline{cpp}{bool} for logical comparisions. In other words: do not use the integer test agains zero. (pp 14)
 \item Bitwise shifts \mintinline{cpp}{<<} and \mintinline{cpp}{>>} \emph{almost} always use zeroes to fill. Except for \emph{signed right shift} which is implementation defined. (pp 15)
 \item Expressions v.s statements. (pp 21)
 \item If we use default value for an argument, then every other argumnet to the \emph{right} of it should also have it. (pp 30)
 \item Use of \mintinline{cpp}{explicit} keyword (pp 32)
 \item Assert as much as you can (pp 35)
 \item C++ can \mintinline{cpp}{throw} anything as exception. Althrougth, is recomended to \mintinline{cpp}{throw} only exceptions. (pp 37)
 \item Always \mintinline{cpp}{catch} an exception by reference. (pp 39)
 \item Catching order matters, and elipsis at the end catch evertything (pp 40)
 \item Generic stream manipulation can be impleneted by: (pp 42)
 \begin{itemize}
  \item Output: \mintinline{cpp}{ostream}
  \item Input: \mintinline{cpp}{istream}
  \item both: \mintinline{cpp}{iostream}
 \end{itemize}
  \item \mintinline{cpp}{Stringstream} can create an \mintinline{cpp}{std::string} from any printable type. (pp 42)
  \item Output stream can formats numbers in octal, decimal and hexadecimal. For binary see this same notes in their bitshift preamble. You can also use \mintinline{cpp}{boolalpha} to format booleans as strings. (pp 44)
  \item Use list initialization with collections, since avoids narrowing. (pp 47)
  \item Use pointer for data structs. (pp 51)
  \item Use RAII when use pointers. (pp 51)
  \item Usage of smart pointers. (pp 51 - 55)
  \item The \mintinline{cpp}{valarray} behaves somehow as a math vector. (pp 58)
  \item Conditional compilation using preprocesor directives. (pp 62)
  \item The preprocessor can be abused to allow nested comments (pp 63)
\end{itemize}

\subsection{Chapter 2}
\begin{itemize}
 \item When to make a constructor \mintinline{cpp}{explicit} (pp 79)
 \item An \mintinline{cpp}{explicit} constructor with more than one argument make sense in C++11 (pp 80)
 \item Delegate contructors: Call another constructor in the initialization argument list of a constructor. (pp 80)
 \item Assigment and copy constructor should be consistent. Assigment should make extra validations against self assigment and compatibility. (pp 81)
 \item An initializer list can be used to construct an object from a list of values of the same type. But we need to implement special assigments and copy constructors. (pp 81-83)
 \item Uniform initializers allow us to use braces \mintinline{cpp}{{ }} as universal initializers, can be dangerous. (pp 83-85)
 \item Use braces in constructor's memeber initialization list, to avoid narrowing (pp 85)
 \item Use move semantics to allow \emph{shallow} copies. (pp 85)
 \item Move semantic is done by providing both: a move copy constructor and a move assigment. (pp 86)
 \item The move constructor steals data from the source and and leaves it in an empty state. Read carefully (pp 86)
 \item Move asigment can be implemented efficientlly by swapping pointer between source and target, (then relying in source's desturctor). Caution: It can can cause unvoluntary swaps if is used with \mintinline{cpp}{std::move()}. (pp 87)
 \item Remember to check compatibillty with move assigment. (Same with normal assigment) (pp 88)
 \item Most move constructor/assigment is called less frequentlly than we expect due \emph{elission}. However, this is compiler dependent and we must not relly on it. (pp 88)
 \item When you use \mintinline{cpp}{std::move()} move semantic its \emph{always} used, but remember an object is considered expired after a move. (pp 89)
 \item Never \mintinline{cpp}{throw} an exception in a destructor. (pp 89)
 \item If a class has at least one \mintinline{cpp}{virtual} member the destructor should also be \mintinline{cpp}{virtual} (pp 89)
 \item Resource allocation it's initialization. (\emph{RAII}). A resource should be tied to an object. I.e it should be allocated in the constructor and released in the destructor. Practically, this means that every time we want to use a resourse we must do it by creating an object that owns it. (pp 90)
 \item Managed resources: we can implement manager clases by using \mintinline{cpp}{std::unique_ptr} and \mintinline{cpp}{std::shared_ptr} only one resource per class. (pp 91)
 \item Release is more important than initialization. Only one single object is responsible for a resource. (pp 91)
 \item A technique for managing resources that depend on other resources is called \emph{resource rescue}. Look at the exmple in (pp 91 - 95)
 \item In C++11, there are six methods with a default behaviour. (pp 95)
 \begin{enumerate}
  \item Default (empthy) constructor.
  \item Copy constructor
  \item Move copy constructor
  \item Assigment operator: \mintinline{cpp}{=}
  \item Assigment move operator: \mintinline{cpp}{=}
  \item Destructor.
 \end{enumerate}
  \item The rule of six: with regard to the six methods mentioned above: Implemnet as little as possible and declare as much as possible. (pp 96)
  \item Member function can return \emph{references} to member data. (pp 96)
  \item Do not keep references of temporary expressions. (pp 97)
  \item To have access to constant data (like in literals) requires a \mintinline{cpp}{const} overload of the method. (pp 97)
  \item You can overload the subscript operator (brakets). If you do, you usually need a \mintinline{cpp}{const} overload too. (pp 99)
  \item For most of the other member functions (methods), only the \mintinline{cpp}{const} version is needed. This is the most common use of \mintinline{cpp}{const}. (pp 99)
  \item The braket (subscript) operator should only apply to an lvalue. To that, we can qualify the member operator with a reference. (pp 100)
  \item We can also reference qualify the asigment operators. (pp 100)
  \item Two ampersands \mintinline{cpp}{&&} can \emph{rvalue} qualify a method. (pp 100)
  \item Define your operators \emph{consitentlly} with each other and whenever appropiate provide a semantic similar to those of an standar type. (pp 101)
  \item You cannot change priority: even for your operator overloads. (pp 102)
  \item Pay attention that the semantic/intended priority of your overloads matches the priority of C++. (pp 102)
  \item Most operators can be defined as member or as a free function. Depending on the case one can make more sense than the other. (pp 102-104)
  \item Implement binary operators as free functions. (pp 104)
  \item Unary operators are \emph{usually} OK as members. (pp 104)
  \item Operators of intrinsic types with primitive types should have conmutative overloads. (pp 103)
  \item Overloading the output stream operator: \mintinline{cpp}{<<}. (pp 104)
  
\end{itemize}
