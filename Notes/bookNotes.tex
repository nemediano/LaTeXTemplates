\section{Notes from book}

While I wasz reading the book~\cite{Gottschling2015}, I had the need to take notes all all the parts that were specially important either because I did not know them or because I am not always aware or their importance (i.e: I tent to forgot them).

\subsection{Chapter 1}
\begin{itemize}
 \item Brace initialization cannot be narrowed (pp 10)
 \item Do not use macros (pp 10)
 \item Addresable is called an \emph{lvale} (pp 13)
 \item Mathemathic expresions should not have side effects. (pp 12)
 \item Always use \mintinline{cpp}{bool} for logical comparisions. In other words: do not use the integer test agains zero. (pp 14)
 \item Bitwise shifts \mintinline{cpp}{<<} and \mintinline{cpp}{>>} \emph{almost} always use zeroes to fill. Except for \emph{signed right shift} which is implementation defined. (pp 15)
 \item Expressions v.s statements. (pp 21)
 \item If we use default value for an argument, then every other argumnet to the \emph{right} of it should also have it. (pp 30)
 \item Use of \mintinline{cpp}{explicit} keyword (pp 32)
 \item Assert as much as you can (pp 35)
 \item C++ can \mintinline{cpp}{throw} anything as exception. Althrougth, is recomended to \mintinline{cpp}{throw} only exceptions. (pp 37)
 \item Always \mintinline{cpp}{catch} an exception by reference. (pp 39)
 \item Catching order matters, and elipsis at the end catch evertything (pp 40)
 \item Generic stream manipulation can be impleneted by: (pp 42)
 \begin{itemize}
  \item Output: \mintinline{cpp}{ostream}
  \item Input: \mintinline{cpp}{istream}
  \item both: \mintinline{cpp}{iostream}
 \end{itemize}
  \item \mintinline{cpp}{Stringstream} can create an \mintinline{cpp}{std::string} from any printable type. (pp 42)
  \item Output stream can formats numbers in octal, decimal and hexadecimal. For binary see this same notes in their bitshift preamble. You can also use \mintinline{cpp}{boolalpha} to format booleans as strings. (pp 44)
  \item Use list initialization with collections, since avoids narrowing. (pp 47)
  \item Use pointer for data structs. (pp 51)
  \item Use RAII when use pointers. (pp 51)
  \item Usage of smart pointers. (pp 51 - 55)
  \item The \mintinline{cpp}{valarray} behaves somehow as a math vector. (pp 58)
  \item Conditional compilation using preprocesor directives. (pp 62)
  \item The preprocessor can be abused for allowing nested comments (pp 63)
  
\end{itemize}
