\chapter{Experimentos}
\label{chap:experimetos}
Para probar la metodología expuesta en el capitulo anterior se realizaron varios experimentos que hemos dividido en dos partes: una con experimentos visuales sobre datos obtenidos de tomógrafos y la segunda parte con experimentos numéricos sobre modelos analíticos. Para todos nuestros experimentos hemos asumido, sin perdida de generalidad, que tenemos volúmenes discretizados $G_{1}$ (en otras palabras $\Delta = 1$). Por lo tanto hemos escalado los blobs de la Sección \ref{sec:repImplicita} de manera consistente (por ejemplo, para el blob diseñado para la rejilla \emph{bcc} el radio es escalado por $\frac{1}{\sqrt{2}}$).

Las propiedades del material son constantes en todos los casos y se proporciona en la Tabla \ref{table:material} en relación al modelo de iluminación. 

\begin{table}[htp]
\begin{center}
  \begin{tabular}{|c|c|c|c|}
    \hline
    $\rho_a$ & $\rho_d$ & $\rho_s$ & $\gamma$ \\
    \hline
    $(0.7, 0.7, 0)$ & $(0.9, 0.9, 0)$ & $(0, 0, 0)$ & $1$  \\
    \hline
  \end{tabular}
\end{center}
\caption[Propiedades del material con el que se visualizan las mallas]{Propiedades del material con el que se visualizan las mallas.}
\label{table:material}
\end{table}
