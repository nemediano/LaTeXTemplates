\chapter{Para ciencias de la computación}
\label{chap:cs}

En éste capitulo, dare algunos tips enfocados a ls ciencias de la computación

Primero voy a mostar como incluir algoritmos en forma de pseudocódigo.
Y desde luego se incluyen en el indice y se pueden referencias asi: El Algoritmo~\ref{alg:euclid} es el primer algoritmo en la historia. Se puede referenciar una linea del algoritmo. El ciclo while termina en la linea line~\ref{euclidendwhile}. 
Las ídeas las tomé de este enlace \href{https://en.wikibooks.org/wiki/LaTeX/Algorithms#Typesetting_using_the_algorithmicx_package}{wikibook} y de éste \href{https://tex.stackexchange.com/questions/229355/algorithm-algorithmic-algorithmicx-algorithm2e-algpseudocode-confused}{post}.

Por defecto el entorno de algoritmo usa todo el ancho de la página.
Es decir, se sale de los márgenes.
Hay un \href{https://tex.stackexchange.com/questions/350434/adjust-width-of-algorithm-float}{truco} para hacerlo entrar en un cierto ancho. Sin embargo, recomiendo usar el truco con moderacion.

{\centering
\begin{minipage}{\linewidth}
  \begin{algorithm}[H]
    \caption{Algoritmo de Euclides}
    \label{alg:euclid}
    \begin{algorithmic}[1] % The number tells where the line numbering should start 0 for no number
      \Procedure{Euclid}{$a,b$} \Comment{El g.c.d. de $a$ y $b$}
        \State $r\gets a \bmod b$
        \While{$r\not=0$} \Comment{Si $r = 0$ ya tenemos la respuesta}
          \State $a \gets b$
          \State $b \gets r$
          \State $r \gets a \bmod b$
        \EndWhile\label{euclidendwhile}
        \State \textbf{return} $b$\Comment{$gcd = b$}
      \EndProcedure
    \end{algorithmic}
  \end{algorithm}
\end{minipage}
\par
}

\section{Código fuente}
Aquí se muestra como incluir código fuente usando el paquete mined.
Este es un ejemplo en el lenguaje C.
\begin{minted}{c}
int main() {
  printf("hello, world");
  return 0;
}
\end{minted}

Este es otro ejemplo de como incluir Python dentro de un parrafo: \mintinline{python}{print(x**2)}.
Finalmente, lo mas util es incluir el codigo fuente desde un archivo externo: Vean  el Listado~\ref{lst:example} como ejemplo.
Me ayude muchisimo de \href{https://tex.stackexchange.com/questions/252263/alignment-of-minted-line-numbers}{aquí} y de la \href{https://www.overleaf.com/learn/latex/Code_Highlighting_with_minted}{ayuda de Overleaf}.
Estamos usando el mismo \textenglish{hack} que usamos con los algoritmos para hacer el listado entrar dentro de los margenes de la página.

{\centering
\begin{minipage}{\linewidth}
  \begin{listing}[H]
  \inputminted[
  xleftmargin=1.5cm,  %without this option line number goes wrong
  %frame=lines,
  framesep=0.5cm,
  baselinestretch=1.2,
  %fontsize=\footnotesize,
  linenos,
  firstline=54, %If you omit this two fields, the whole file is pulled
  lastline=68
  ]{cpp}{src/GccTest.cpp}
  \caption{Mi implementacion erronea de selection sort. (No la uses tiene un error)}
  \label{lst:example}
  \end{listing}
\end{minipage}
\par
}
