\chapter*{Agradecimientos}

{\small
Agradezco al Consejo Nacional de Ciencia y Tecnología (\textsc{conac}y\textsc{t}), por apoyar esta investigación con la beca \textsc{cvu}--266004 para estudios de Maestría en la Universidad Nacional Autónoma de México.

La tesis fué realizada bajo la supervisión de Edgar Garduño, quien me invito a participar en este proyecto. Agradezco enormemente su apoyo, guía y paciencia.

Estoy agradecido con mis sinodales: Jorge Urrutia, Fernando Arámbula, Jorge Márquez y Ernesto Bribiesca, por las importantes recomendaciones, sugerencias y correcciones. Sin duda hicieron que este trabajo mejorara su calidad. 

Agradezco también al Departamento de Ciencias de la Computación del Instituto de Investigaciones en Matemáticas Aplicadas y en Sistemas por permitirme hacer uso de sus instalaciones y por todos los recursos materiales con los que me apoyaron.

Y por supuesto, gracias\ldots

A mi mamá (q. e. p. d.), mi papá y mis hermanos, por siempre haber creído en mi y por haberme apoyado no solo en este trabajo, si no en todos los proyectos que he emprendido en la vida.

A Erika, por las platicas, el apoyo, los momentos compartidos, la solidaridad y por ser la persona tan maravillosa que es.

A Magali, Laura y Jose Luis, mis \emph{roomates}, por haberme permitido compartir su vida y su amistad.

A mi grupo de trabajo: Etna, Fátima, Eduardo, Verena y Cinthya. Por la compañia, por haberme rebotado ideas y por siempre haber estado dispuestos a ayudarme.

A \emph{los inges}: Sergio, Federico y Omar, por haber sido mis compañeros y amigos. Y por haberme apoyado en esos momentos tan difíciles que me tocó pasar. Agradezco también mis compañeros de clase durante la maestría. En especial a Tzolkin, Carlos Alegría, Eliza, Uriel, Marlene, Nahela, Paty, Jaime Cabrera, Toño, Sebastian Bejos y Sonia. Por haberme permitido estudiar a su lado, por todo lo que les aprendí y por su amistad.

A mi amigos: José Raul, Sergio (maría), Jorge (primo), Bernardo, Onatta, Yesenia, Karina, Mariano y Chac. Primero por su amistad, luego por todas las veces que me escucharon y por último por haber tenido siempre tiempo de preguntarme: \emph{¿Cómo va tu tesis?}

A Carmen Villar, por su amistad y por haberme mostrado por primera vez el maravilloso mundo de las gráficas por computadora.

A toda la gente que paga sus impuestos y que cumple con su trabajo en paz.
}