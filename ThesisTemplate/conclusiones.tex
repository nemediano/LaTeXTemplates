\chapter*{Conclusiones}
\addcontentsline{toc}{chapter}{Conclusiones}
\markboth{CONCLUSIONES}{CONCLUSIONES}

Hay que señalar que los resultados son puramente visuales y altamente dependientes del modelo de iluminación. Aunque en este trabajo nos limitamos al modelo de iluminación de Phong y al sombreado de Gouraud, bien valdría la pena explorar modelos alternativos tanto para iluminación como para sombreado.

También como trabajo futuro se podría intentar que el GPU se hiciera cargo no solo de los cálculos de iluminación si no también de los cálculos de extracción de superficie, es decir que el Algoritmo de Artzy debería de poderse implementar en GPU.

La elección del blob Kaiser-Bessel para la construcción de la superficie implícita fue en gran medida al conocimiento previo del grupo de trabajo y a que ha dado buenos resultados para fines de visualización. Sin embargo, este blob es relativamente complejo de evaluar. Por esto no se descarta la posibilidad de usar algún otro blob y que pueda dar resultados parecidos y con mucho menos costo computacional.

Tradicionalmente se han usado técnicas de procesamiento de imágenes para mejorar la visualización de imágenes en biomedicina. Los resultados de éste trabajo demuestran que es posible mejorar la visualización de una imagen 3D usando técnicas de iluminación. Por lo tanto, debe de seguirse investigando como mejorar las imágenes  no solo con procesamiento de imágenes, si no también por técnicas de graficación por computadora. 
