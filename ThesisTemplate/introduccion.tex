\chapter*{Introducción}
\addcontentsline{toc}{chapter}{Introducción}
\markboth{INTRODUCCIÓN}{INTRODUCCIÓN}

La percepción que tenemos los seres humanos del mundo a través de nuestros sentidos es inmediata. Ademas, dependemos de estos sentidos para realizar la mayoría de las actividades diarias. También hay que señalar que somos criaturas visuales, de nuestros cinco sentidos el que nos proporciona la mayor cantidad de la información que asimilamos sin lugar a dudas son nuestros ojos \cite{Hagen:vista}.

Desde hace mas de tres décadas las computadoras se han vuelto mas poderosas y hemos tenido la oportunidad de procesar con ellas una mayor cantidad de datos de manera rápida. Una de las áreas que se ha encontrado mas beneficiada es la imagenología biomédica. Si bien es cierto que los rayos X eran desde hace mucho una herramienta probada para el diagnostico médico (y diversas aplicaciones industriales), una imagen digital que pudiera almacenarse, procesarse y luego desplegarse en una computadora ha sido un parteaguas para este campo. Al mismo tiempo, muchas áreas de las ciencias de la computación (CC) han cobrado interés durante el mismo periodo. Particularmente, el procesamiento de imágenes digitales, las gráficas por computadora y la visualización científica, son ahora una opción viable para usarse rutinariamente.

El procesamiento de imágenes estudia como procesar imágenes digitales por medios computacionales \cite{Gonzalez:ImagenesDigitales}. Algunas veces este proceso implica transformar una imagen en otra imagen en donde se ha resaltado, o inhibido, cierta característica. Otras veces implica obtener otro tipo de información de una imagen, por ejemplo un histograma. Finalmente, algunas veces implica poder distinguir o interpretar datos de objetos a partir de las imágenes.

Las gráficas por computadora (GC) son la rama de las CC que estudia las técnicas para producir imágenes digitales a partir de descripciones matemáticas. Estas imágenes son visualizadas en dispositivos de despliegue en 2D, por ejemplo un monitor, por lo que generalmente es necesario hacer una proyección de los objetos (en tres dimensiones) a un plano.

La visualización científica se encarga de construir información visual de conjuntos de datos científicos. Estos datos pueden tener orígenes muy diversos; inclusive pueden simular fenómenos de la naturaleza que no pueden ser percibidos por nuestra vista. Por ejemplo, ver en la pantalla el espacio de solución de una ecuación diferencial que modele la vibración de la cuerda de una guitarra es un problema de visualización científica que se podría equiparar con la idea de ``visualizar la música''.

En éste trabajo se abarca un poco de estas tres áreas. Asumimos que tenemos un conjunto de datos distribuidos en un arreglo cúbico que representan un muestreo de un campo escalar tridimensional. Estos datos forman lo que típicamente se conoce como una imagen digital en 3D o volumen. Debido a la naturaleza de los datos nos interesa poder visualizarlos sin perder su información espacial.

%La manera como haremos esto es realizando una segmentación que nos permita extraer una superficie de la imagen y luego pretendemos visualizar esa superficie. Lo primero es encontrar una malla que se aproxime a la superficie. Para esto se hace uso de un algoritmo de extracción de superficies propuesto por Herman y Artzy en \cite{ArtzyLargo}. Este algoritmo tiene la desventaja de obtener una aproximación a las superficie formada por caras de cuadrados orientadas ortogonalmente. Por esta razón los resultados se ven como si fueran bloques de LEGO\textcopyright.
Una forma de lograr éste objetivo es encontrar una superficie que defina la frontera de un objeto y luego visualizar la superficie como si ésta fuera un envolvente. Como en un principio se tiene la imagen en 3D es necesario utilizar algún criterio que defina la superficie (proceso típicamente conocido como segmentación). Dado la naturaleza del volumen debemos encontrar una manera de generar un conjunto de polígonos para aproximar la superficie de manera digital.

En este trabajo estudiaremos dos métodos para aproximar esas superficies: el algoritmo de \emph{Marching Cubes} \cite{MarchingCubes}, que es el mas usado actualmente, y el Algoritmo de Artzy \cite{artzyCorto}. El Algoritmo de Artzy tiene varias ventajas sobre el primero, la mas importante es que garantiza que la superficie aproximada es topológicamente correcta, es decir que no contiene agujeros. Sin embargo, la desventaja mas importante de Artzy es que las mallas obtenidas parecen cuboides, como si fueran hechas por bloques del juego de LEGO.%$^\copyright$.

Sin importar el método de obtención de la malla, se visualiza por medio de técnicas de graficación por computadora. Esto incluye utilizar \emph{iluminación}, que en conjunto con otras técnicas, tiene la función de dar la apariencia de 3D a la proyección en 2D. Para poder utilizar la iluminación es necesario que proporcionemos un conjunto de vectores normales a la superficie en cada vértice de la malla. Como es mostrado en la técnica conocida como \emph{bump mapping} \cite{bussCG}, estos vectores pueden afectar significativamente la apariencia final de la imagen.

En este trabajo proponemos una forma de obtener un conjunto de vectores normales en los vértices de la malla obtenida por el Algoritmo de Artzy, tales que al usarse en conjunto con la iluminación, el resultado final sea una apariencia mas suave del objeto. Por lo tanto se obtendría una malla que conserva las ventajas del Algoritmo de Artzy pero que sea visualmente equiparable con la malla producida por \emph{Marching Cubes}. Para esto, pensamos encontrar un envolvente suave de la malla por medio de funciones base de transición suave de uno a cero; conocidas como Kaiser Bessel generalizadas.

La organización del trabajo es la siguiente. En el primer capítulo, se hace la fundamentación matemática necesaria de lo que entendemos como volumen y se explican que condiciones esperamos que cumplan los volúmenes de los que hablamos en el resto del trabajo. También se analizan los dos algoritmos de rastreo de superficies sobre volúmenes antes expuestos poniendo énfasis en las propiedades presentes en el Algoritmo de Artzy. También se revisa el modelo de iluminación de Phong y se habla de los modelos de sombreado de Phong y de Gouraud. Por último se revisan dos técnicas de GC que usan mapas para dar efectos: el mapeo de texturas y el mapeo de relieves (\emph{bump mapping}).

En el segundo capítulo se introducen las superficies implícitas y algunas funciones base que comúnmente se usan para crear dichas superficies. Luego se introducen las funciones Kaisser Bessel generalizadas que servirán de funciones base en el resto del trabajo. También se exponen las contribuciones del trabajo. La primera es que se sigue el método de optimización para funciones base propuesto en \cite{EdgarOptimization} y se llegan a parámetros óptimos para tres casos. La segunda aportacion es que se obtiene un algorimo para crear la superficie implícita que envuelva la malla producida por el Algoritmo de Artzy y como se le asignan normales a los vértices de esa malla. Esta última es la principal aportación del trabajo.

El tercer capítulo esta dedicado a los experimentos realizados para probar la metodología expuesta. En primer lugar se muestran resultados puramente visuales sobre algunos conjuntos de datos provenientes de imagenología biomédica. Posteriormente, se explica la manera como se construyen dos \emph{phantoms} (un cubo y una esfera) a diferentes resoluciones y se evalúan las diferencias entre las normales analíticas de los \emph{phantoms} y las obtenidas por nuestro método.

Por último, en el capítulo final se presentan las conclusiones del trabajo y conjuntamente se proponen lineas de investigación futuras.

\subsubsection*{Objetivos del trabajo}

El principal objetivo de este trabajo es encontrar por medio de la modificación de normales una forma de hacer la superficie obtenida por el Algoritmo de Artzy visualmente mas agradable. Haciendo enfasis en que no hacemos ninguna suposición sobre como fue la discretización del volumen.